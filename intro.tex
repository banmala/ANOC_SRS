\pagenumbering{roman}
% \large
% 	\chapter*{Certificate of Approval}
% \normalsize 
% \addcontentsline{toc}{section}{Certificate of Approval}
% 	The undersigned certify that the sixth semester project entitled \textbf{“Voice Assistant for Visually Impaired People”} submitted by Anusha Bajracharya, Luja Shakya, Niranjan Bekoju and Sunil Banmala to the \textbf{Department of Computer Engineering} in partial fulfillment of requirement for the degree of \textbf{Bachelor of Engineering in Computer Engineering}. The project was carried out under special supervision and within the time frame prescribed by the syllabus. \\ 
% 	\\
% 	We found the students to be hardworking, skilled, bona fide and ready to undertake any commercial and industrial work related to their field of study and hence we recommend the award of Bachelor of Computer Engineering degree. \\
% 	\vspace{2cm} \\
% 	.............................................\\
% 	Er. Ramesh Marikhu\\
% 	(Project Supervisor)\\
% 	\vspace{1cm}\\
% 	.............................................\\
% 	Er. Shiva Kumar Shrestha\\
% 	Head of Department \\
% 	Department of Computer Engineering, KhCE\\
% \pagebreak



% \large
% 	\chapter*{Copyright}
% \normalsize
% \addcontentsline{toc}{section}{Copyright}
% 	The author has agreed that the library, Khwopa College of Engineering  may make this report freely available for inspection. Moreover, the author has agreed that permission for the extensive copying of this project report for scholarly purpose may be granted by supervisor who supervised the project work recorded here in or, in absence the Head of The Department where in the project report was done. It is understood that the recognition will be given to the author of the report and to Department of Computer Engineering, KhCE in any use of the material of this project report. Copying or publication or other use of this report for financial gain without approval of the department and author’s written permission is prohibited. Request for the permission to copy or to make any other use of material in this report in whole or in part should be addressed to: \\
% 	\vspace{1cm} \\
% 	Head of Department \\
% 	Department of Computer Engineering\\
% 	Khwopa College of Engineering\\
% 	Liwali,\\
% 	Bhaktapur, Nepal\\
% \pagebreak


% \large
% \chapter*{Acknowledgement}
% \normalsize
% \addcontentsline{toc}{section}{Acknowledgement}
% We take this opportunity to express our deepest and sincere gratitude to our HoD Er. Shiva Kumar Shrestha, for his insightful advice, motivating suggestions for this project and also for his constant encouragement and advice throughout our Bachelor’s program.
% \\We are deeply indebted to our DHoD Er. Dinesh Gothe for boosting our efforts and moral by his valuable advises and suggestion regarding the project, giving us realization of the practical scenario of the project and supporting us in tackling various difficulties.\\Also, we would like to thank Er. Bindu Bhandari for providing valuable suggestions and for supporting the project.
% \begin{table}[h]
% 	\begin{tabular}{ll}
% 		Anusha Bajracharya & KCE074BCT011 \\
% 		Luja Shakya        & KCE074BCT022 \\
% 		Niranjan Bekoju    & KCE074BCT025 \\
% 		Sunil Banmala      & KCE074BCT045 \\
% 	\end{tabular}
% \end{table}
% \pagebreak

% \large
% \chapter*{Abstract}
% \normalsize
% \addcontentsline{toc}{section}{Abstract}
% Whenever landowner wants to build a house, he needs to prepare design (floor plan) of the house. He needs to decide where the main entrace, opening will be, how is he going to split the room, what portion of buildings will be seperated for bedroom, kitchen, bathroom etc. These are general questions that hits in mind. In order to solve these queries, he consult an architect. Architect would use different planning tools to generate the plan of the building. Initially, it would be difficult for an architect to make plan. So, author introduced Floor Plan Generation using GAN that would produce conceptual floor plan that best suits parcel of the land to provide a vision that can help architects. Architects would be able to choose among generated plan and then modify accordingly. This method would be relatively easier than directly generating plan from scratch. Moreover, to generate the plan, the system will get parcel of the land from architect, mapped it to footprint, room split and finally furnished room. The system will use conditional GAN for generation. It will also generate the 3D model of generated floor plan. 
% \\\\
% \textbf{Keywords}:\textit{Condtional GAN, U-Net architecuture, 3D model generation}

\pagebreak

\tableofcontents
% \addcontentsline{toc}{section}{Contents}

% \listoftables
% \addcontentsline{toc}{section}{List of Tables}
% \break
% \pagebreak

% \listoffigures
% \addcontentsline{toc}{section}{List of Figures}
% \pagebreak

% \listoftables
% \addcontentsline{toc}{section}{List of Tables}
% \pagebreak

% \Large
% \begingroup
% \let\clearpage\relax
% \chapter*{List of Abbreviation}
% \endgroup
% \normalsize
% \addcontentsline{toc}{section}{List of Abbreviation}

% \begin{table}[h]
% 	\begin{tabular}{l l}
% 		\textbf{Abbreviations} & \textbf{Meaning}                                               \\
% 		AI                     & Artificial Intilligence                                        \\
% 		ASPP                   & Atrous Spatial Pyramid Pooling                                 \\
% 		cGAN                   & Conditional Generative Adversarial Network                     \\
% 		CV                     & Computer Vision                                                \\
% 		CVC-FP                 & Computer Center Vision - Floor Plan                            \\
% 		DNN                    & Deep Neural Network                                            \\
% 		FID                    & Frechet Inception Distance                                     \\
% 		FCN                    & Fully Convolutional Network                                    \\
% 		GAN                    & Generative Adversial Network                                   \\
% 		GC-LPN                 & Graph Conditioned - Layout Prediction Network                  \\
% 		ISCC-NBS               & Inter-Society Color Council-National Bureau of Standards       \\
% 		LCT-GAN                & Language Conditioned Textures - Generative Adversarial Network \\
% 		LOFD                   & Local Orientation and Frequency Description                    \\
% 		OCR                    & Optical Character Recognition                                  \\
% 		R-FP                   & Request for Proposal                                           \\
% 		ROBIN                  & Repository Of BuildIng plaNs                                   \\
% 		SUGAMAN                & Supervised and Unified framework using Grammar and             \\
% 		                       & Annotation Model for Access and Navigation                     \\
% 		VAE-GAN                & Variational Autoencoder- Generative Adversarial Network        \\
% 		VCN                    & Volumetric Convolutional Network                               \\
% 		VGG net                & Visual Geometry Group Network                                  \\
% 	\end{tabular}
% \end{table}
% \pagebreak

